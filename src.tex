\documentclass[czech, kiv, ba, he, iso690alph, pdf]{fasthesis}
\title{\texttt{FASThesis} -- návod k~použití šablony závěrečné práce na FAV ZČU}
\author{Leonid}{Malakhov}{}{}

\assignment{zadani.pdf}
\signdate{31}{12}{2024}{V Nové Vsi u~Nového Města na Moravě}% <== the longest local name in the Czech Rep.




\begin{document}
	\tableofcontents


	\chapter{Úvod}
	Tato \term{šablona kvalifikační práce} nastavuje převážnou většinu parametrů sazby v~typografickém systému \LaTeX{} tak, aby výsledný dokument odpovídal jednak požadavkům norem (pokud existují) a jednak tradičním zvyklostem úpravy vědecko-technických dokumentů. Je navržena tak, aby nebylo nutné žádné parametry sazby \uv{ručně} upravovat. Zkušení uživatelé \LaTeX{}u nicméně pochopitelně mohou vzhled sazby snadno změnit použitím příkazů \LaTeX{}u a maker, která poskytuje základní třída dokumentu využitá v~této šabloně. Zejména z~tohoto důvodu -- pro zkušené uživatele -- je tedy dobré vědět, že
	\begin{center}
		\framebox{třída \filename"fasthesis" této šablony je založena na třídě \filename"memoir".}
	\end{center}
	
	Šablona ovlivňuje zásadním způsobem vzhled výsledného dokumentu tak, aby jednak jasně deklaroval příslušnost k~Fakultě aplikovaných věd Západočeské univerzity v~Plzni\footnote{Šablona je k~dispozici jako open source, a je tedy možné ji libovolně modifikovat za podmínky dodržení ustanovení licence LGPL. Pokud tedy nejste z~FAV ZČU a chcete tuto šablonu použít, směle do toho: Bude zřejmě třeba změnit barevnost a zejména \textbf{font!} Proč? Použitý bezpatkový font nadpisů \emph{GT America} není volně k~dispozici -- ZČU je vlastníkem komerční licence na jeho použití, neboť je nedílnou součástí jejího vizuálního stylu. Jste-li tedy z~jiné instituce, která nemá tento font zakoupený, použít jej nemůžete a je \textbf{nutné} jej nahradit jiným (doporučuji třeba volně dostupný font \emph{Roboto Condensed}).}, druhak aby byl text přehledný a dobře čitelný a aby bylo pokud možno \uv{o~vše postaráno} automaticky a autor nemusel řešit např. vzhled a provedení citací použité literatury, umístění a přesné znění prohlášení atp.
	
	\section{Instalace}
	Šablonu není nutné  na váš počítač vůbec instalovat (stejně jako celý \TeX) v~případě, že hodláte k~sepsání kvalifikační práce využít online editor \term{Overleaf} -- na adrese URL \url{https://www.overleaf.com/read/ryhpnsmtgrrs} je šablona pro tento editor již k~dispozici, připravena k~okamžitému použití. Pro studenty FAV ZČU se pak nabízí ještě lepší možnost, a to využít lokální instalace Overleafu/Share\LaTeX{}u na serveru Katedry informatiky a výpočetní techniky FAV ZČU na adrese URL \url{https://overleaf.kiv.zcu.cz} -- tam se totiž neuplatňuje omezení na čas kompilace dokumentu jako v případě zdarma dostupného Overleafu.
	
	Pokud ovšem používáte \TeX{} pravidelně a máte jej tedy nainstalovaný na svém počítači (což je silně \textbf{doporučená} varianta), není instalace šablony také nijak komplikovaná. Postupujte podle krok za krokem podle níže uvedeného návodu:
\end{document}